\documentclass[9pt, a5paper]{book}


\usepackage[utf8]{inputenc}
\usepackage[T1]{fontenc}
\usepackage[english, russian]{babel}
\usepackage{amsmath}
\usepackage{relsize}
\usepackage[left=1cm,right=1.5cm,top=1cm,bottom=0.5cm,bindingoffset=0cm]{geometry}
\linespread{0.9}



\begin{document}

\setcounter{page}{166}

\noindent из (3) и подставляя в (2), ищем ненулевое решение $a_k$ ($k = 1, 2,{}\dots{}, n$) \linebreak
системы (2). Подставляя найденные значения $a_k$ в (1), получаем при- \linebreak
ближенное выражение собственной функции, отвечающей найденному \linebreak
собственному значению. \\

{\bfseries Пример 1.} Найти по методу Ритца приближенное значение наимень- \linebreak
\indent шего характеристического числа ядра.
$$K(x, t) = xt; \qquad a = 0, b = 1.$$

\underline {\textsl {Решение}}. В качестве координатной системы функции $\psi_n(x)$ выбираем \linebreak
систему полиномов Лежандра: $\psi_n(x) = P_n(2x - 1)$. В формуле (1) ограничимся
двумя слагаемыми, так что
$$\psi_2(x) = a_1 \cdot P_0(2x - 1) + a_2 \cdot P_1(2x - 1)$$
Замечая, что
$$\psi_1 \equiv P_0(2x - 1) = 1; \qquad \psi_2 \equiv P_1(2x - 1) = 2x - 1,$$
находим
$$(\psi_1, \psi_1) = \int\limits^1_0 dx = 1, \qquad (\psi_1, \psi_2) = (\psi_2, \psi_1) = \int\limits^1_0 (2x - 1)dx = 0,$$
$$(\psi_2, \psi_2) = \int\limits^1_0 (2x-1)^2dx = \dfrac{1}{3}.$$
Далее,
$$(K\psi_1, \psi_1) = \int\limits^1_0 \left(\int\limits^1_0 K(x, t) \psi_1 (t) dt \right)\psi_1(x)dx = \int\limits^1_0 \int\limits^1_0xt\,dx\,dt = \dfrac{1}{4},$$
\qquad \ \, $\displaystyle{(K\psi_1, \psi_2) = \int\limits^1_0 \int\limits^1_0 xt(2x - 1)dx\,dt = \dfrac{1}{12},}$
\vspace{\baselineskip} \par
\noindent \qquad \ \, $\displaystyle{(K\psi_2, \psi_2) = \int\limits^1_0 \int\limits^1_0 xt(2t - 1)(2x - 1)dx\,dt = \dfrac{1}{36},}$

\vspace{\baselineskip}
Система (3) в этом случае принимает вид

\begin{equation*}
   \begin{vmatrix} 
   \dfrac{1}{4} - \sigma & \dfrac{1}{12} \\
   \\
   \dfrac{1}{12} & \dfrac{1}{36}-\dfrac{1}{3}\sigma
   \end{vmatrix} 
   = 0,
\end{equation*}
или
$$\sigma^2 - \sigma\left(\dfrac{1}{12} + \dfrac{1}{4}\right) = 0.$$
\newpage
\setcounter{page}{167}

\noindent Отсюда $\sigma_1 = 0, \, \sigma_2 = \dfrac{1}{3}.$ Наибольшее собственное значение $\sigma_2 = \dfrac{1}{3},$ значит, \linebreak
наименьшее характеристическое число $\lambda = \dfrac{1}{\sigma_2} = 3$.

\vspace{\baselineskip}
\noindent {\bfseries \Large Задачи для самостоятельного решения}

\noindent\rule{13cm}{0.4pt}

\vspace{\baselineskip}
\noindent По методу Ритца найти наименьшие характеристические числа ядер $(a = 0, \\ b = 1)$:

\begin{equation*}
\mathlarger {\boldsymbol{340.}} \, K(x, t) = x^2 t^2. \qquad \mathlarger {\boldsymbol{341.}} \, K(x, t) = 
 \begin{cases}
   t, & x \geq t, \\
   x, & x \leq t.
 \end{cases}
\end{equation*}

\begin{equation*}
\mathlarger {\boldsymbol{342.}} \, K(x, t) =
 \begin{cases}
   \frac{1}{2}x(2-t), & x \leq t, \\
   \frac{1}{2}t(2-x), & x \geq t.
 \end{cases}
\end{equation*}

\noindent\rule{13cm}{0.4pt}

\vspace{\baselineskip}
\begin{center}
{\bfseries \Large $2^{o}$. Метод следов.} Назовём m-м следом ядра $K(x, t)$ число
\end{center}
$$A_m = \int\limits^b_a K_m(t,t)\,dt,$$
Где $K_m(x,t)$ означает m-е итерированное ядро.

Для наименьшего характеристического числа $\lambda_1$ при достаточно \linebreak
большом m справедлива следующая приближенная формула:
$$|\lambda_1| \approx \sqrt{\frac{A_{2m}}{A_{2m+2}}}.\eqno(4)$$ 
\noindent Формула (4) дает значение $|\lambda_1|$ с избытком.

Следы четного порядка для симметричного ядра вычисляются по формуле

$$A_{2m} = \int\limits^b_a \int\limits^b_a K^2_m(x,t)\,dx\,dt=2 \quad \int\limits^b_a \int\limits^b_a K^2_m(x,t)\,dt\,dx.\eqno(5)$$

{\bfseries Пример 2.} Найти по методу следов первое характеристическое число \linebreak
ядра

\begin{equation*}
 K(x,t)=
 \begin{cases}
   t, & x \geq t, \\
   x, & x \leq t,
 \end{cases}
 \quad a = 0, b = 1.
\end{equation*}

\end{document}
